%! Author = teismar
%! Date = 9/17/24
%! File = Grundcharakter.tex
\cleardoublepage
\section{Kapitel 2}
\label{sec:kap2}

Hier ist ein weiteres Beispiel für eine Tabelle, die den Vergleich zwischen traditionellen und flexiblen Unternehmensmodellen verdeutlicht. Diese Tabelle zeigt die Unterschiede in Struktur, Entscheidungsfindung, Kommunikation, Arbeitsteilung, Mitarbeiterrolle, Fokus sowie Stärken und Schwächen der beiden Modelle.
\begin{table}[H]
    \centering
    \caption{Vergleich traditioneller und flexibler Unternehmensmodelle}
    \label{tab:vergleich_modelle}
    \begin{tabular}{|p{4cm}|p{5cm}|p{5cm}|}
        \hline
        \textbf{Merkmal} & \textbf{Traditionelles Modell} & \textbf{Flexibles Modell} \\ \hline
        Struktur & Hierarchisch, starr & Flach, anpassungsfähig \\ \hline
        Entscheidungsfindung & Top-down & Dezentral, teamorientiert \\ \hline
        Kommunikation & Formal, vertikal & Offen, horizontal \\ \hline
        Arbeitsteilung & Streng, spezialisiert & Flexibel, teamübergreifend \\ \hline
        Mitarbeiterrolle & Ausführend & Mitgestaltend, eigenverantwortlich \\ \hline
        Fokus & Effizienz, Stabilität & Agilität, Innovation \\ \hline
        Stärken & Klare Verantwortlichkeiten, Stabilität in vorhersehbaren Umgebungen & Anpassungsfähigkeit, Innovationskraft, Mitarbeiterzufriedenheit \\ \hline
        Schwächen & Starrheit, langsame Entscheidungsfindung, geringe Mitarbeitermotivation & Hoher Koordinationsbedarf, erfordert hohe Eigenverantwortung der Mitarbeiter \\ \hline
    \end{tabular}
\end{table}